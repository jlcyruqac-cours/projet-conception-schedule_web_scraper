\documentclass[12pt]{article}
\usepackage[ruled, linesnumbered, french, onelanguage]{algorithm2e}
\usepackage[utf8]{inputenc}
\usepackage[T1]{fontenc}
\usepackage{amsmath, amsthm, amssymb, amsfonts}
\usepackage{vmargin}
\usepackage{setspace}
\usepackage{graphicx}
\usepackage[colorinlistoftodos]{todonotes} % add todos
\usepackage{indentfirst} % indent first line of paragraph
\usepackage{hyperref} % Add hyper ref
\usepackage{tabularx} % For table auto line break
\usepackage{scrextend}
\usepackage[frenchb]{babel}
\addto\captionsfrench{\def\tablename{Tableau}}
\usepackage{float} % here for H placement parameter
\usepackage[shortlabels]{enumitem} % dash to bullet for item
\usepackage{enumitem} % dash to bullet
\usepackage{pdflscape} % Turn page to landsscape
\usepackage{vhistory} % historique
\usepackage{multirow} % multirow des tables
\usepackage{subfiles} % Use to separate .tex into section .tex file

\newenvironment{changemargin}[2]{%
\begin{list}{}{%
\setlength{\topsep}{0pt}%
\setlength{\leftmargin}{#1}%
\setlength{\rightmargin}{#2}%
\setlength{\listparindent}{\parindent}%
\setlength{\itemindent}{\parindent}%
\setlength{\parsep}{\parskip}%
}%
\item[]}{\end{list}}


\setlist[itemize]{label=\textbullet} % dash to bullet for item
\graphicspath{{figures/}}

\usepackage{mathtools}

\usepackage{fancyhdr}

\title{Plan de projet}

\date{\today}
\newcommand{\version}{1.0}

\makeatletter
\let\thetitle\@title
\let\thedate\@date
\makeatother

\pagestyle{fancy}
\fancyhf{}
\lhead{\thetitle { \version}}
\rhead{\thedate}
\cfoot{\thepage}

\begin{document}
\begin{titlepage}


    \newcommand{\HRule}{\rule{\linewidth}{0.5mm}} % Defines a new command for the horizontal lines, change thickness here
     
    %----------------------------------------------------------------------------------------
    %   LOGO SECTION
    %----------------------------------------------------------------------------------------
    \centering  
    \includegraphics[width=6cm]{uqac-logo} \\[2cm] % Include a department/university logo - this will require the graphicx package
     
    %---------------------------------------------------------------------------------------- 


    \center % Center everything on the page 
     
    %----------------------------------------------------------------------------------------
    %   HEADING SECTIONS
    %----------------------------------------------------------------------------------------
    
    \textsc{\LARGE Génie informatique} \\[1.5cm] % Name of your university/college
    \textsc{\Large Applications réseaux et sécurité informatique} \\[0.5cm] % Major heading such as course name
    \textsc{\large 6GEI466} \\[1cm] % Minor heading such as course title
    
    %----------------------------------------------------------------------------------------
    %   TITLE SECTION
    %----------------------------------------------------------------------------------------
    
    \HRule \\[0.4cm]
    \Large \bfseries Projet de conception \\
    \Large Application d'horaire de cours \\ % Title of your document
    \HRule \\[2cm]
     
    %----------------------------------------------------------------------------------------
    %   AUTHOR SECTION
    %----------------------------------------------------------------------------------------
   
  	\begin{onehalfspacing}
    \begin{table}[H]
    \begin{tabular}{l l}

        \large \emph{Préparé par :} & \quad \quad \quad \large \emph{Pour:} \\
    	Alissa \textsc{Bonnel} & \quad \quad \quad Monsieur Jean-Luc \textsc{Cyr}, \\
        Jean-Sébastien \textsc{St-Pierre} & \quad \quad \quad \textsc{Université du Québec à Chicoutimi} \\
        Alexis \textsc{Valotaire} \\
        \textsc{génie informatique}  \\ \\ \\ \\
    
    \end{tabular}
	\end{table}
	\end{onehalfspacing}
   

    %----------------------------------------------------------------------------------------
    %   DATE SECTION
    %----------------------------------------------------------------------------------------

    {\large \today} % Date, change the \today to a set date if you want to be precise
    
    \vfill % Fill the rest of the page with whitespace


\end{titlepage}


\newpage
\section*{Historique des versions}
\begin{table}[H]
    \begin{tabular}{|l|l|l|p{7.6cm}|}
        \hline
        \textbf{Version} & \textbf{Date} & \textbf{Auteur(s)} & \textbf{Modifications} \\ \hline
        1.0 & 04-11-2019 & AB, JSSP, AV & Création du document \\ \hline
    \end{tabular}
\end{table}


\section*{Définitions}
\begin{table}[H]
\begin{tabular}{|l|p{12.5cm}|}
\hline
\textbf{Terme} & \textbf{Définition} \\ \hline

\textit{Framework}       &   Ensemble cohérent de composants éprouvés et réutilisables ainsi que
de préconisations servant à créer les fondations et les grandes lignes
d'un logiciel ou de certains de ses composants \cite{defFramework}. \\ \hline


\textit{Front-end}       &   Partie du développement applicatif web visant à implémenter les
mécanismes d'interaction avec l'utilisateur, habituellement sous forme
d'interfaces graphiques. Le Front-End s'exécute le plus souvent sur un
ordinateur client et communique avec un serveur traitant les requêtes et
les données \cite{defServerless}. \\ \hline


\textit{Back-end}       &   Partie du développement applicatif web visant à implémenter les
mécanismes de traitement, de logique et d'accès aux données, qui est
habituellement exécutée sur un serveur. Le Front-End et le Back-End d'une même
application communiquent entre eux au moyen d'un protocole de
communication réseau \cite{defServerless}. \\ \hline


\textit{Cloud server}      &   Terme désignant une architecture faisant appel à une
infrastructure infonuagique où un serveur virtuel, dont les ressources peuvent
être allouées dynamiquement par le fournisseur de services ("pay-as-you-go"),
est utilisé en lieu et place d'un serveur physique disposant de
ressources  statiques \cite{defServerless}. \\ \hline


\end{tabular}
\end{table}


\section*{Abréviations et acronymes}
\begin{table}[H]
\begin{tabular}{|l l|}
\hline
\textbf{Abréviation/acronyme} & \textbf{Définition} \\  \hline
SPA & Single page application \\
BCAPG & Bureau canadien d'agrément des programmes de génie \\ \hline
\end{tabular}
\end{table}

\newpage

\tableofcontents

\listoffigures

\listoftables

\newpage
    
\section{Introduction}

\subsection{Vue d'ensemble du projet}

\subsection{Références}

\subsection{Portée}

\subsection{Livrables}

\newpage

\section{Organisation du projet}

\section{Stratégie de contrôle des versions}

\section{Directives de livraison}

\newpage

\section{Ressources matérielles, calendrier et budget}

\subsection{Ressources matérielles}

\subsection{Calendrier des taches du projet}

\subsection{Budget}

\newpage

\section{Gestion des risques}

\subsection{Risque :}

\subsubsection*{Description :}

\subsubsection*{Probabilité :}

\subsubsection*{Conséquence :}

\subsubsection*{Exposition :}

\subsubsection*{Méthodes de contention :}


\end{document}