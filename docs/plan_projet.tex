\documentclass[12pt]{article}
\usepackage[ruled, linesnumbered, french, onelanguage]{algorithm2e}
\usepackage[utf8]{inputenc}
\usepackage[T1]{fontenc}
\usepackage{amsmath, amsthm, amssymb, amsfonts}
\usepackage{vmargin}
\usepackage{setspace}
\usepackage{graphicx}
\usepackage[colorinlistoftodos]{todonotes} % add todos
\usepackage{indentfirst} % indent first line of paragraph
\usepackage{hyperref} % Add hyper ref
\usepackage{tabularx} % For table auto line break
\usepackage{scrextend}
\usepackage[frenchb]{babel}
\addto\captionsfrench{\def\tablename{Tableau}}
\usepackage{float} % here for H placement parameter
\usepackage[shortlabels]{enumitem} % dash to bullet for item
\usepackage{enumitem} % dash to bullet
\usepackage{pdflscape} % Turn page to landsscape
\usepackage{vhistory} % historique
\usepackage{multirow} % multirow des tables
\usepackage{subfiles} % Use to separate .tex into section .tex file

\newenvironment{changemargin}[2]{%
\begin{list}{}{%
\setlength{\topsep}{0pt}%
\setlength{\leftmargin}{#1}%
\setlength{\rightmargin}{#2}%
\setlength{\listparindent}{\parindent}%
\setlength{\itemindent}{\parindent}%
\setlength{\parsep}{\parskip}%
}%
\item[]}{\end{list}}


\setlist[itemize]{label=\textbullet} % dash to bullet for item
\graphicspath{{figures/}}

\usepackage{mathtools}

\usepackage{fancyhdr}

\title{Plan de projet}

\date{\today}
\newcommand{\version}{1.0}

\makeatletter
\let\thetitle\@title
\let\thedate\@date
\makeatother

\pagestyle{fancy}
\fancyhf{}
\lhead{\thetitle { \version}}
\rhead{\thedate}
\cfoot{\thepage}

\begin{document}
\begin{titlepage}


    \newcommand{\HRule}{\rule{\linewidth}{0.5mm}} % Defines a new command for the horizontal lines, change thickness here
     
    %----------------------------------------------------------------------------------------
    %   LOGO SECTION
    %----------------------------------------------------------------------------------------
    \centering  
    \includegraphics[width=6cm]{uqac-logo} \\[2cm] % Include a department/university logo - this will require the graphicx package
     
    %---------------------------------------------------------------------------------------- 


    \center % Center everything on the page 
     
    %----------------------------------------------------------------------------------------
    %   HEADING SECTIONS
    %----------------------------------------------------------------------------------------
    
    \textsc{\LARGE Génie informatique} \\[1.5cm] % Name of your university/college
    \textsc{\Large Applications réseaux et sécurité informatique} \\[0.5cm] % Major heading such as course name
    \textsc{\large 6GEI466} \\[1cm] % Minor heading such as course title
    
    %----------------------------------------------------------------------------------------
    %   TITLE SECTION
    %----------------------------------------------------------------------------------------
    
    \HRule \\[0.4cm]
    \Large \bfseries Projet de conception \\
    \Large Application de suivi des mises à jour des locaux et horaires de cours à l'UQAC \\ % Title of your document
    \HRule \\[2cm]
     
    %----------------------------------------------------------------------------------------
    %   AUTHOR SECTION
    %----------------------------------------------------------------------------------------
   
  	\begin{onehalfspacing}
    \begin{table}[H]
    \begin{tabular}{l l}

        \large \emph{Préparé par :} & \quad \quad \quad \large \emph{Pour:} \\
    	Alissa \textsc{Bonnel} & \quad \quad \quad Monsieur Jean-Luc \textsc{Cyr}, \\
        Jean-Sébastien \textsc{St-Pierre} & \quad \quad \quad \textsc{Université du Québec à Chicoutimi} \\
        Alexis \textsc{Valotaire} \\
        \textsc{génie informatique}  \\ \\ \\ \\
    
    \end{tabular}
	\end{table}
	\end{onehalfspacing}
   

    %----------------------------------------------------------------------------------------
    %   DATE SECTION
    %----------------------------------------------------------------------------------------

    {\large \today} % Date, change the \today to a set date if you want to be precise
    
    \vfill % Fill the rest of the page with whitespace


\end{titlepage}


\newpage
\section*{Historique des versions}
\begin{table}[H]
    \begin{tabular}{|l|l|l|p{7.6cm}|}
        \hline
        \textbf{Version} & \textbf{Date} & \textbf{Auteur(s)} & \textbf{Modifications} \\ \hline
        1.0 & 04-11-2019 & AB, JSSP, AV & Création du document \\ \hline
    \end{tabular}
\end{table}


\section*{Définitions}
\begin{table}[H]
\begin{tabular}{|l|p{12.5cm}|}
\hline
\textbf{Terme} & \textbf{Définition} \\ \hline

\textit{Framework}       &   Ensemble cohérent de composants éprouvés et réutilisables ainsi que
de préconisations servant à créer les fondations et les grandes lignes
d'un logiciel ou de certains de ses composants \cite{defFramework}. \\ \hline


\textit{Front-end}       &   Partie du développement applicatif web visant à implémenter les
mécanismes d'interaction avec l'utilisateur, habituellement sous forme
d'interfaces graphiques. Le Front-End s'exécute le plus souvent sur un
ordinateur client et communique avec un serveur traitant les requêtes et
les données \cite{defFrontBackEnd}. \\ \hline


\textit{Back-end}       &   Partie du développement applicatif web visant à implémenter les
mécanismes de traitement, de logique et d'accès aux données, qui est
habituellement exécutée sur un serveur. Le Front-End et le Back-End d'une même
application communiquent entre eux au moyen d'un protocole de
communication réseau \cite{defFrontBackEnd}. \\ \hline


\textit{Cloud server}      &   Terme désignant une architecture faisant appel à une
infrastructure infonuagique où un serveur virtuel, dont les ressources peuvent
être allouées dynamiquement par le fournisseur de services ("pay-as-you-go"),
est utilisé en lieu et place d'un serveur physique disposant de
ressources  statiques \cite{defServerless}. \\ \hline

\textit{Web scraping}      &   Terme désignant un ensemble de techniques permettant l'extraction rapide de grandes quantités de données rendues disponibles par un site web, le plus souvent dans le code HTML.   \cite{scraping}. \\ \hline


\end{tabular}
\end{table}


\section*{Abréviations et acronymes}
\begin{table}[H]
\begin{tabular}{|l l|}
\hline
\textbf{Abréviation/acronyme} & \textbf{Définition} \\  \hline
SPA & Single page application \\
BCAPG & Bureau canadien d'agrément des programmes de génie \\ \hline
\end{tabular}
\end{table}

\newpage

\tableofcontents

\listoffigures

\listoftables

\newpage
    
\section{Introduction}

\subsection{Vue d'ensemble du projet}
Le projet consiste au développement d'une application web permettant à l'utilisateur d'accéder
aux locaux et horaires de cours rapidement et facilement.  Dans le contexte quelque peu frénétique et, plus récemment, cahotique des débuts de trimestres à l'UQAC, il est parfois difficile pour les étudiants et enseignants de demeurer à l'affût des inévitables changements logistiques qui peuvent être apportés par l'administration sans aucun préavis.  L'application qui sera réalisée dans le cadre de ce projet vise à apporter une solution simple à cette problématique récurrente en permettant une détection automatique des changements, qui pourront être notifiés à l'usager selon ses préférences.  \\

Concrètement, l'utilisateur pourra se créer un compte et y associer ses cours.  Une extraction (\textit{scraping}) sera effectuée périodiquement sur le contenu des pages web correspondantes du site institutionnel de l'UQAC dans le but de détecter tout changement apporté aux horaires et numéros de locaux.  L'utilisateur aura la possibilité de s'authentifier, puis de consulter son horaire (incluant les numéros des locaux).  Il pourra également choisir d'être notifié en temps réel de tout changement par le moyen de son choix (courriel, SMS, notification de l'application, etc.).

\subsection{Références}
Les références utilisées dans le cadre de ce rapport peuvent être consultées dans la bibliographie. 

\subsection{Portée}
L'application sera utilisable par toute personne impliquée dans les activitées de formation de l'UQAC (étudiants, professeurs, chargés de cours, etc.)  L'application ne vise en aucun temps à se substituer au portail officiel de l'institution; elle se veut plutôt une forme d'extension l'offre existante. \\

Le bon fonctionnement de l'application sera dépendant de l'exactitude des informations publiées sur le site officiel de 
l'UQAC.  Vu le recours à la technique du \textit{scraping}, des correctifs devront lui être apportés suite à tout changement dans la structure des documents web de l'institution.

\subsection{Livrables}
Lors de la complétion du projet, la documentation de l'application sur son fonctionnement et son 
déploiement sera fournie ainsi que le code source. 

\newpage

\section{Organisation du projet}

\subsection{Composition de l'équipe du projet}

Notre équipe se compose de trois étudiants en génie informatique à leur dernière
année du baccalauréat: \\
    
% TODO Separation des roles, juste competence de liste

\noindent \textbf{Alexis Valotaire} \\
Compétences/Préférences :
\begin{itemize}
    \item Maîtrise du Python
    \item Connaissance en HTML, CSS, JavaScript
    \item Connaissances en design web
    \item Connaissance en configuration et maintenance de serveurs
    \item Connaissance en gestion de base de données
    \item Connaissance en réseautique
    \item Initié en industrie aux bases de l’industrie 4.0, IoT, cloud
    computing, etc \\
\end{itemize}

\noindent \textbf{Jean-Sébastien St-Pierre} \\
Compétences/Préférences :
\begin{itemize}
    \item Maîtrise des langages C/C++
    \item Connaissances en Python et Javascript
    \item Connaissances en systèmes de gestion de bases de données et SQL
    \item Connaissances en conception de bases de données
    \item Connaissances en réseautique
    \item Initié en industrie aux bases de l’industrie 4.0, IoT, cloud
    computing, etc. \\
\end{itemize}

\noindent \textbf{Alissa Bonnel} \\
Compétences/Préférences :
\begin{itemize}
    \item Maîtrise des langages C/C++, PHP
    \item Connaissances en Python et Web 
\end{itemize}

\subsection{Stratégie de contrôle des versions}

Le contrôle des versions se fera avec le système de gestion de version qu'est GitHub autant
pour la documentation que pour le code du projet.

\subsection{Directives de livraison}

\newpage

\section{Ressources matérielles, calendrier et budget}

\subsection{Ressources matérielles}

Aucune ressources matérielles physiques ne sera nécessaire. Seul un serveur infonuagique sera nécessaire
afin d'héberger l'application.

\subsection{Calendrier des taches du projet}

\subsection{Budget}

\newpage

\section{Gestion des risques}

% TODO Ajouter les risques trouves dans le devoir?

\subsection{Risque :}

\subsubsection*{Description :}

\subsubsection*{Probabilité :}

\subsubsection*{Conséquence :}

\subsubsection*{Exposition :}

\subsubsection*{Méthodes de contention :}

\newpage 

% bibliography
\def\bibindent{1em}
\begin{thebibliography}{99\kern\bibindent}
\makeatletter
\let\old@biblabel\@biblabel
\def\@biblabel#1{\old@biblabel{#1}\kern\bibindent}
\let\old@bibitem\bibitem
\def\bibitem#1{\old@bibitem{#1}\leavevmode\kern-\bibindent}
\makeatother

    
    \bibitem{defFramework}
    Formation Django. (s. d.) Qu'est-ce qu'un framework?. [En ligne].
    Disponible: \url{http://www.formation-django.fr/generalites/framework.html}

    \bibitem{defFrontBackEnd}
    CONCEPTANEXT. (2019) What Is the Difference Between Front-End and Back-End Development?. [En ligne].
    Disponible: \url{https://conceptainc.com/blog/difference-front-end-back-end-development/}
    
    \bibitem{defServerless}
    Cloudflare. (s. d.) What is serverless computing?. [En ligne]. Disponible:
    \url{https://www.cloudflare.com/learning/serverless/what-is-serverless}

    \bibitem{scraping}                   
    Kho, J. (2018) How to Web Scrape with Python in 4 Minutes. [En ligne]. Disponible:
    \url{https://towardsdatascience.com/how-to-web-scrape-with-python-in-4-minutes-bc49186a8460}

    \bibitem{uCalgary}                  
    University of Calgary. (2015) Energy education - Hydroelectric reservoir.
    [En ligne]. Disponible: \url{https://energyeducation.ca/encyclopedia/
    Hydroelectric_reservoir}
    
    \bibitem{kundur}
    P. Kundur, \textit{Power System Stability and Control}, New York, États-Unis
    : McGraw-Hill, 1994.

    \bibitem{hChute}                    
    CTEC. (2007) Analyse de projets d'énergie propre: MManuel d'ingénierie et d'études de cas RETScreen®.
    [En ligne]. Disponible: \url{http://publications.gc.ca/collections/collection_2007/nrcan-rncan/M39-98-2003F.pdf}

    \bibitem{barrage}
    Tennessee Valley Authority, version SVG par Tomia (adaptation française).
    (2010) File:Hydroelectric dam.svg. [En ligne]. Disponible: \url{https://
    commons.wikimedia.org/wiki/File:Hydroelectric_dam.svg}. Image sous licence
    libre CC-BY-2.5 \url{https://creativecommons.org/licenses/by/2.5/legalcode}

    \bibitem{sSeguinLogiOptim}                  
    Séguin, S. “Conception d'un logiciel d'optimisation pour la production
    hydroélectrique”. Département d’informatique et de mathématique, Université
    du Québec à Chicoutimi, Chicoutimi, Québec, Rapport technique, 2019.
\end{thebibliography}

\end{document}